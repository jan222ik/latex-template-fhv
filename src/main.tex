% Preamble
\documentclass[11pt, twoside]{report}

% Packages

\usepackage[utf8]{inputenc}
\usepackage[T1]{fontenc}                            %Sets fontenc to max level needed in file
\usepackage{graphicx}                               %To load and display images
\usepackage{caption}                                %For subfigures and cap env
\usepackage{subcaption}
\usepackage[a4paper, width=150mm,                   %Size of Paper
            top=25mm, bottom=25mm,
            bindingoffset=6mm]{geometry}
\usepackage{fancyhdr}                               %Header and Footers
\pagestyle{fancy}                                   %To disable for specific page use \pagestyle{<type>}
                                                    %<type>: plain for pagenr
                                                        %    empty for empty pagestyle
                                                        %    fancy to enable the default syle again
\fancyhead{}                                        %Resets the header
\fancyhead[RO,LE]{\chaptername~\thechapter}         %Header Definition: Chaptername and number: Right on Odd and Left on even pages
\fancyfoot{}                                        %Resets the footer
\fancyfoot[CE, CO]{\thepage}                        %Centered Page Number on odd and even pages
%Define output dir, because of error with: https://tex.stackexchange.com/questions/112953/error-when-using-minted-package-and-output-directory-option
%Cache can be removed normally, but in certain situations it can result in errors. Setup minted see: https://alipourmousavi.com/blog/index.php/2018/02/08/using-minted-package-in-latex-to-format-codes/
\usepackage[cache=false, outputdir=Z:/Users/jan22/CodeProjects/latex_template/auxil]{minted} %Must be before csquotes
\setminted{
    linenos=true,
    autogobble
}
\usepackage[english]{babel}                         %For splitting words
\usepackage{csquotes}                               %Used for babel/polyglossia and minted
\usepackage{setspace}                               %Set line spaceing
\setstretch{1.5}
\usepackage{amsmath}                                %Math package
\usepackage{color}                                  %Colored Links
\usepackage{hyperref}                               %For setting links
\hypersetup{
    colorlinks=false,                                %true, if links should be colorful
    linktoc=all,                                    %all => sections and subsections linked
    linkcolor=black,                                %Color of link
}
\usepackage{listings}                               %Environment for listings
%\usepackage{eurosym}                               %Euro sign
%Maybe change back to biber
\usepackage[style=authoryear,citestyle=authoryear,backend=bibtex]{biblatex}   % Bib
\addbibresource{main.bib}                           %Set bib file

%FOR minted
\renewcommand{\MintedPygmentize}{Z:/Users/jan22/AppData/Local/Programs/Python/Python37-32/Scripts/pygmentize} %Set path to script location
\renewcommand{\theFancyVerbLine}{\sffamily\textcolor[rgb]{0.5,0.5,0.5}{\scriptsize\arabic{FancyVerbLine}}}
% Create a new environment for breaking code listings across pages.
%Change lstlisting to figure if code fragments should be listed under figures
\newenvironment{longlisting}{\captionsetup{type=lstlisting}}{}
\renewcommand{\lstlistingname}{Code Fragment}       %Sets name of Listing space lstlisting - Will occured before each caption inside its environment
\renewcommand{\lstlistlistingname}{List of \lstlistingname s} %Sets title for \listoflstlistings




%CUSTOM VARS:
\newcommand{\submissionDay}{day-of-submission}
\newcommand{\submissionMonth}{month-of-submission}
\newcommand{\submissionYear}{year-of-submission}
\newcommand{\maintitle}{thesis-title}
\newcommand{\subtile}{thesis-title-addition}
\newcommand{\name}{author-title-name}
\newcommand{\supervisor}{supervisor-title-name}

% Document
\begin{document}
    % Titelblatt:
% \newpage\mbox{}\newpage
\cleardoublepage   % force output to a right page
\thispagestyle{empty}
\begin{titlepage}
    \begin{flushright}
        \includegraphics[width=0.4\linewidth]{../src/other-pages/title/Logo-A3.jpg}
    \end{flushright}
    %! Suppress = LineBreak
    \begin{flushleft}
        \vspace{0.5cm}

        \section*{\maintitle}
        %\subsection*{\subtile}

        \vspace{0.5cm}

        Bachelor Thesis 2\\
        In Partial Fulfillment
        of the Requirements for the Degree of
        Bachelor of Science

        \vspace{0.5cm}

        \textbf{Bachelor of Science in Software and Information Engineering}

        \vspace{1cm}
        University of Applied Sciences FH Vorarlberg\newline
        Software and Information Engineering

        \vspace{0.5cm}

        Supervised by:\newline
        \phantom{x}\hspace{3ex}\supervisor\newline
        \vspace{0.5cm}

        Submitted by:\newline
        \phantom{x}\hspace{3ex}\name\newline
        \newline
        Dornbirn, \submissionMonth\space\submissionYear
    \end{flushleft}
\end{titlepage}

\newpage
\addtocontents{toc}{\protect\setcounter{tocdepth}{-1}}
\chapter*{Abstract}\label{ch:abstract}
Text

\chapter*{AbstractGER}\label{ch:abstractger}
Text

\addtocontents{toc}{\protect\setcounter{tocdepth}{2}}
\newpage
%Table of contents
\tableofcontents
\renewcommand{\thechapter}{\Roman{chapter}}
\newpage

\setcounter{chapter}{0}
\addcontentsline{toc}{chapter}{\protect\numberline{I}List Of Abbreviations}
\chapter*{List Of Abbreviations}
\clearpage
\begingroup %Prevent with a group the creation of new pages of \listof<name>
\let\clearpage\relax
\addcontentsline{toc}{chapter}{\protect\numberline{II}List Of Figures}
\listoffigures
\addcontentsline{toc}{chapter}{\protect\numberline{III}List Of Tables}
\listoftables
\addcontentsline{toc}{chapter}{\protect\numberline{IV}List Of Code Fragments}
\lstlistoflistings
\endgroup
\cleardoublepage

%Content
\renewcommand{\thechapter}{\arabic{chapter}}
\setcounter{chapter}{0}




\chapter{Template Chapter}\label{ch:template-chapter}
\section{Images}\label{sec:images}
See Image~\ref{fig:example-picture} on Page~\pageref{fig:example-picture}:
\begin{figure}[hbt!]
    \centering
    \includegraphics[scale=1.0]{../src/images/screenshot.png}
    \caption{An example picture}
    \label{fig:example-picture}
\end{figure}

Multiple Figures in one environment:
Use hfill for margin between for multi line leave blank lines.
\begin{figure}[hbt!]
    \centering
    \begin{subfigure}[hbt!]{0.49\textwidth}
        \centering
        \includegraphics[width=\textwidth]{../src/images/screenshot.png}
        \caption{Picture One}
        \label{fig:one}
    \end{subfigure}
    \hfill
    \begin{subfigure}[hbt!]{0.49\textwidth}
        \centering
        \includegraphics[width=\textwidth]{../src/images/screenshot.png}
        \caption{Picture Two}
        \label{fig:two}
    \end{subfigure}
    \newline
    \begin{subfigure}[hbt!]{\textwidth}
        \centering
        \includegraphics[width=\textwidth]{../src/images/screenshot.png}
        \caption{Picture Three}
        \label{fig:three}
    \end{subfigure}
    \caption{Three pictures}
    \label{fig:three-pictures}
\end{figure}

Reference each fig inside~\ref{fig:three-pictures} like:\newline
One:~\ref{fig:one}\newline
Two:~\ref{fig:two}\newline
Three:~\ref{fig:three}\newline
\clearpage
\section{Tables}\label{sec:tables}
I can reference the table by~\ref{tab:basic-table}
\begin{table}[hbt!]
    \centering
    \begin{tabular}{|l|l|l|}
        A & B & C \\
        \hline
        1 & 2 & 3
    \end{tabular}
    \caption{A basic table}
    \label{tab:basic-table}
\end{table}
See for subtables use:
\begin{table}[hbt!]
    \begin{subtable}[hbt!]{0.45\textwidth}
        \centering
        \begin{tabular}{|l|l|l|}
            A & B & C \\
            \hline
            1 & 2 & 3
        \end{tabular}
        \caption{Multi table 0}
        \label{tab:multi-table-0}
    \end{subtable}
    \hfill
    \begin{subtable}[hbt!]{0.45\textwidth}
        \centering
        \begin{tabular}{|l|l|l|}
            A & B & C \\
            \hline
            1 & 2 & 3
        \end{tabular}
        \caption{Multi table 1}
        \label{tab:multi-table-1}
    \end{subtable}
    \caption{Two tables in one line}
    \label{tab:multi-table}
\end{table}

\section{Minted Code Fragments}\label{sec:minted-code-fragments}

\begin{longlisting}
    \begin{minted}{java}
        public class WaterKettleSteps implements cucumber.api.java8.En {
            @Inject
            public WaterKettleSteps(WaterPage page) {
                When("I waited for \"{int}\" minutes my water is boiling",
                (Integer min) -> {
                    page.assertWaterBoilingAfter(min);
                });
            }
        }

        public class WaterPage {
            public void assertWaterBoilingAfter(Integer min) {
                if (min < 5) { fail(); }
            }
        }
    \end{minted}
    \caption[Short Code Example]{A shorter code example, which will not break across pages.}
    \label{lst:short}
\end{longlisting}

\section{Citation}\label{sec:citation}

Use citation of entry with key rose\textunderscore cucumber\textunderscore 2015: \newline
Citation without brackets with cite\textbraceleft rose\textunderscore cucumber\textunderscore 2015\textbraceright\space like~\cite{rose_cucumber_2015}. \newline
Citation with brackets with parencite\textbraceleft rose\textunderscore cucumber\textunderscore 2015\textbraceright\space like~\parencite{rose_cucumber_2015}.

\input{chapters/ipsumChapter.tex}


\cleardoublepage
\renewcommand{\thechapter}{\Roman{chapter}}
\setcounter{chapter}{3}
%\pagenumbering{Roman}
\appendix
\addcontentsline{toc}{chapter}{\protect\numberline{V}Appendix}
\chapter*{Appendix}
%Bibliography
\setcounter{biburllcpenalty}{7000} %Adds breakpoints in URLs - https://tex.stackexchange.com/questions/134191/line-breaks-of-long-urls-in-biblatex-bibliography
\setcounter{biburlucpenalty}{8000}
\addcontentsline{toc}{chapter}{\protect\numberline{VI}Bibliography}
\printbibliography

%Eigenständigkeitserklärung
%\input{documents/content/resources/other-pages/6_eidesstatlicheerklaerung.tex}

\end{document}
